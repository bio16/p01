\section{Ejercicio 3}

En esta secci\'on consdideramos una red (ver archivo \texttt{as-22july06.gml}) creada por Mark Newman que contiene la estructura de los sistemas aut\'onomos de internet relevada a mediados del 2006.


\subsection{Parte a}

Primero hallamos todos los grados $k$ de la red, usando la rutina \texttt{igraph.Graph.Read\_GML()} de la libreria \texttt{igraph} (version 0.7.1) de Python.
La distribuci\'on $P_k$ de dichos valores de grado se observa en la figura \ref{fig:ej03_1}, donde claramente observamos un fuerte comportamiento tipo power-law, a lo largo de tr\'es d\'ecadas en $k$.
Exploramos visualmente la tendencia de $P_k$ en otras escalas lineales y semi-logar\'itmicas, pero no fueron las m\'as apropiadas para caracterizarla.


\begin{figure}
\centering
\includegraphics[scale=.5]{figuras/ej03_Pk}
\caption{Distribuci\'on de probabilidad $P_k$ de los valores de grado $k$ de la red de Mark Newman.
Observamos que la distribuci\'on tiene fuerte comportamiento tipo power-law, especialmente en el rango $k \sim (3-400)$.}
\label{fig:ej03_1}
\end{figure}



\subsection{Parte b}
